\section{Code source SQL}

Voici comment nous créons nos entités en SQL :
\begin{enumerate}
\item[•] Utilisateur : \\
\lstinputlisting[language=SQL, firstline=1, lastline=10]{source.sql}
\item[•] Client : \\
\lstinputlisting[language=SQL, firstline=21, lastline=29]{source.sql}
\item[•] Commande : \\
\lstinputlisting[language=SQL, firstline=31, lastline=40]{source.sql}
\item[•] Famille : \\
\lstinputlisting[language=SQL, firstline=42, lastline=46]{source.sql}
\item[•] Sous-Famille : \\
\lstinputlisting[language=SQL, firstline=48, lastline=54]{source.sql}
\item[•] Article : \\
\lstinputlisting[language=SQL, firstline=56, lastline=62]{source.sql}
\item[•] Composition : \\
\lstinputlisting[language=SQL, firstline=64, lastline=71]{source.sql}
\end{enumerate}

\noindent
\textit{Si nous choisissons finalement de créer une entité Role, on aura :}\\
\lstinputlisting[language=SQL, firstline=12, lastline=18]{source.sql}