\section{Modèle logique}

En utilisant le modèle conceptuel de nos données nous en déduisons le modèle logique suivant :
\begin{enumerate}
\item[•] Utilisateur(\underline{id}, nom, prenom, mail, password, role*)
\item[•] Client(\underline{code}, adresse, code postal, ville, \#id\_utilisateur)
\item[•] Commande(\underline{id}, date, lieu, margeTotal, chiffre d’affaires, \#id\_utilisateur)
\item[•] Composition(\underline{\#id\_commande, \#id\_article}, marge)
\item[•] Article(\underline{id}, dateAchat, \#codeFam\_famille)
\item[•] Famille(\underline{codeFam}, libFam)
\item[•] SousFamille(\underline{codeSfam}, libSfam, \#codeFam\_famille) \\
\end{enumerate}

\noindent
\textit{* : Nous nous réservons la possibilité de supprimer ce paramètre d'Utilisateur et de créer à la place l'entité : Role(\underline{id}, nomRole, \#id\_utilisateur)} \\ \\

Notons par ailleurs que nous pourrions effectuer les modifications suivantes :
\begin{enumerate}
\item[•] Modifier la cardinalité de la relation entre un article et une famille telle que article(1,1)-->famille.
\item[•] Décomposer l'adresse du client. En effet dans un souci de normalisation nous pourrions la diviser en un numéro et un nom de rue.
\end{enumerate}